\PassOptionsToPackage{unicode=true}{hyperref} % options for packages loaded elsewhere
\PassOptionsToPackage{hyphens}{url}
\PassOptionsToPackage{dvipsnames,svgnames*,x11names*}{xcolor}
%
\documentclass[]{article}
\usepackage{lmodern}
\usepackage{amssymb,amsmath}
\usepackage{ifxetex,ifluatex}
\usepackage{fixltx2e} % provides \textsubscript
\ifnum 0\ifxetex 1\fi\ifluatex 1\fi=0 % if pdftex
  \usepackage[T1]{fontenc}
  \usepackage[utf8]{inputenc}
  \usepackage{textcomp} % provides euro and other symbols
\else % if luatex or xelatex
  \usepackage{unicode-math}
  \defaultfontfeatures{Ligatures=TeX,Scale=MatchLowercase}
\fi
% use upquote if available, for straight quotes in verbatim environments
\IfFileExists{upquote.sty}{\usepackage{upquote}}{}
% use microtype if available
\IfFileExists{microtype.sty}{%
\usepackage[]{microtype}
\UseMicrotypeSet[protrusion]{basicmath} % disable protrusion for tt fonts
}{}
\IfFileExists{parskip.sty}{%
\usepackage{parskip}
}{% else
\setlength{\parindent}{0pt}
\setlength{\parskip}{6pt plus 2pt minus 1pt}
}
\usepackage{xcolor}
\usepackage{hyperref}
\hypersetup{
            pdftitle={Pay for Bits: Probably a Better Method for Rewarding Forecasters},
            pdfauthor={Nuño Sempere},
            colorlinks=true,
            linkcolor=Maroon,
            filecolor=Maroon,
            citecolor=Blue,
            urlcolor=blue,
            breaklinks=true}
\urlstyle{same}  % don't use monospace font for urls
\setlength{\emergencystretch}{3em}  % prevent overfull lines
\providecommand{\tightlist}{%
  \setlength{\itemsep}{0pt}\setlength{\parskip}{0pt}}
\setcounter{secnumdepth}{0}
% Redefines (sub)paragraphs to behave more like sections
\ifx\paragraph\undefined\else
\let\oldparagraph\paragraph
\renewcommand{\paragraph}[1]{\oldparagraph{#1}\mbox{}}
\fi
\ifx\subparagraph\undefined\else
\let\oldsubparagraph\subparagraph
\renewcommand{\subparagraph}[1]{\oldsubparagraph{#1}\mbox{}}
\fi

% set default figure placement to htbp
\makeatletter
\def\fps@figure{htbp}
\makeatother


\title{Pay for Bits: Probably a Better Method for Rewarding Forecasters}
\author{Nuño Sempere\footnote{Quantified Uncertainty Research Institute.}}
\date{\today}

\begin{document}
\maketitle

\hypertarget{motivation}{%
\section{Motivation}\label{motivation}}

In \href{https://arxiv.org/abs/2106.11248}{Alignment Problems With
Current Forecasting Platforms}, Sempere and Lawsen outline a variety of
problems with current forecasting platforms, whose scoring rules are
found to either not be proper---as in the case of Good Judgment Open or
CSET-Foretell (now INFER)---or incentivize distorting one's true
probabilities to maximize the chances of placing in the top few
positions which earn a monetary reward---as in the case of Metaculus. In
addition, in almost all cases, forecasting platforms---or, for that
matter, prediction markets---disincentivize collaboration.

In this working paper, we outline an incentivization method, ``paying
for bits'', which combines features of prediction markets and
forecasting platforms to produce a scoring rule that pays forecasters
well, is proper, and nonetheless incentivizes collaboration. This
scoring rule has both a discrete form---where resolution is either a yes
or a no, and a continuous form---where resolution can be any probability
between 0 and 1.

In essence, we start out with
\href{https://mason.gmu.edu/~rhanson/mktscore.pdf}{Hansons' logarithmic
market scoring rule}, and we add a few bells and whistles to make it
collaborative. Although interacting with a prediction market run by an
automatic market maker is known to be equivalent to interacting with a
scoring rule, we prefer the former interpretation, which is a better
abstraction because it deals more elegantly with forecasters not being
able to input arbitrarily high or low probabilities.

In particular, the practical innovations we suggest are:

\begin{itemize}
\tightlist
\item
  Instead of comparing forecasters against other forecasters, compare
  them against against the initial probability of the stakeholder who
  sponsors the question
\item
  Create a prediction market for each forecaster with a logarithmic
  market scoring rule, and allow each forecaster to move only the
  probability of their own prediction market.
\item
  Require forecasters to put their money where their mouth is by betting
  their own money, so that if their forecast is worse that the houses'
  own forecast, they loose money. Optionally, sponsor forecasters once
  who start out without many fund themselves.
\end{itemize}

In two accompanying papers, Amplify Samotsvety and Amplify Rootclaim, we
use the continuous form of this scoring rule like a lego-block: we
combine it with other methods to build more powerful and general
incentive schemes.

\hypertarget{a-few-limitations-of-previous-approaches-and-their-solutions}{%
\section{A few limitations of previous approaches, and their
solutions}\label{a-few-limitations-of-previous-approaches-and-their-solutions}}

\hypertarget{market-scoring-rules-overcome-the-limitations-of-brier-and-logarithmic-scoring-rules}{%
\subsection{Market scoring rules overcome the limitations of Brier and
logarithmic scoring
rules}\label{market-scoring-rules-overcome-the-limitations-of-brier-and-logarithmic-scoring-rules}}

The Brier score has limitations stemming from its deep theoretical
inelegance. Sure, it is proper, but it also doesn't have desirable
properties such as:

\begin{itemize}
\tightlist
\item
  Composability: The Brier score on ``A'' and the Brier score on ``B
  given A'' can't be straightforwardly related to the Brier score on ``A
  and B''
\item
  Comparability: The cumulative relative Brier score doesn't correspond
  to any particular meaning.
\item
  Comparability II: The Brier score of the average guess will always be
  at least as good as the average of the Brier scores, and usually
  better. This incentivizes forecasters to forecast the average crowd
  guess.
\end{itemize}

In contrast, the logarithmic scoring rule has an array of desirable
properties.

\begin{itemize}
\item
  Composability: The logarithmic score on ``A\&B'' is the sum of the log
  scores on ``A'' and ``B given A''
\item
  Comparability: The cumulative relative log score corresponds to bits
  of information which would be added or removed
\item
  Comparability II: The arithmetic mean of the log scores is the log
  score of the geometric mean of the guesses. If forecasters are
  rewarded in proportion to their log score vs the crowd, there is no
  advantage to forecasting the current crowd.

  More general, for many nice theoretical properties, the log score will
  have them, whereas the Brier score will not. And yet, the Brier score
  offers bounded payoffs, whereas the logarithmic score does not. That
  is, when distributing a fixed pot of money as rewards to forecasters,
  the natural way to do this would be in proportion to their scores. But
  in the case of the logarithmic scoring rule, it is not bounded, which
  would complicate payoff calculations.\footnote{Readers may think that
    normalization by the worst logarithmic score may solve this. But
    because the worst score can be arbitrarily low, preserving incentive
    compatibility is not trivial.}
\end{itemize}

This is why platforms such as Cultivate Labs still use the Brier score
to reward their members. In contrast, platforms such as Metaculus solve
this problem by forbidding their members from inputting probabilities
lower than 1\% or higher than 99\%.

However, a logarithmic market scoring rule can approximate the
logarithmic score, while bounding payoffs by only allowing players to
loose as much as they have previously accumulated. Conversely, the more
money---or points in the case of a play-money prediction market---a
participant has accumulated, the more extreme they are allowed to make
the market. This preserves some of the desirable properties of the
logarithmic scoring rule in approximate form, while making payoffs
bounded without having to institute a sharp cut-off.

From a theoretical perspective and pointers to previous literature,
Hanson's \href{https://mason.gmu.edu/~rhanson/mktscore.pdf}{Logarithmic
Market Scoring Rules for Modular Combinatorial Information Aggregation}
is a dry if comprehensive introduction. From the perspective of a
programmer seeking to implement a logarithmic market scoring rule,
Gnosis'
\href{https://gnosis-pm-js.readthedocs.io/en/v1.3.0/lmsr-primer.html}{LMSR
Primer} is particularly readable.

To summarize, a logarithmic scoring rule is given by
\(s_i = a_i + b\cdot log(p_i)\), where \(p_i\) is the probability the
player assigns to an event. We can rewrite this as
\(s_i = b \cdot log(p_i / c_i)\), so that it becomes clear that \(b\)
determines how much the automatic market maker pays or receives per some
amount of improvement of information over its prior, \(c_i\).
Intuitively, \(b\) should be higher the more the market sponsor cares
about the event.

Then, from that scoring rule, we can create an equivalent market scoring
rule. A market scoring rule is determined by a cost function, \(C\),
such that the cost of buying an additional \(n\) shares of outcome \(i\)
is given by \(C(q_i + n)-C(q_i)\), where \(q_i\) are the number of
shares for that outcome already issued. In the case of the logarithmic
market marker, the cost function is given by:

\[C(\vec{q}) = b \cdot log\left( \sum_i{exp\left(\frac{q_i + a_i}{b}\right)}\right)\]

where \(a_i\) is given by the original credences of the market
creator\footnote{Most discussions of the LMSR omit the $a_i$ parameter. This might be because the only ones who bothered to do the math were [Pennock](http://blog.oddhead.com/2006/10/30/implementing-hansons-market-maker/) and [Chenn and Pennock](https://arxiv.org/abs/1206.5252), and they likewise omit this parameter.}:

\[a_i = b \cdot log\left(\frac{1}{c_i}\right)\]

Then the marginal price is given by

\[p_i = \frac{\partial C}{\partial q_i} = \frac{exp\left(\frac{q_i + a_i}{b}\right)}{\sum_j{exp\left(\frac{q_j + a_j}{b}\right)}}\]

Note that from a platform design perspective, \(\vec{a}\) can be
abstracted away by setting \(\vec{q}=\vec{a}\), i.e., by initializing a
market maker with a number of shares already issued. Note also that
players cannot loose more than the account in their balance, or,
conversely, that the more they are willing to risk, the more extreme
their probabilities can be. This seems more elegant than discrete
cutoffs at \(<1\%\) or \(>99\%\), as in Metaculus.

\hypertarget{competing-against-the-house-provides-an-incentive-for-collaboration-in-a-way-which-competing-against-other-forecasters-doesnt.}{%
\subsection{Competing against the house provides an incentive for
collaboration in a way which competing against other forecasters
doesn't.}\label{competing-against-the-house-provides-an-incentive-for-collaboration-in-a-way-which-competing-against-other-forecasters-doesnt.}}

To determine the relative quality of forecasters, forecasting platforms
currently score and compare forecasters in either of two ways:

\begin{itemize}
\tightlist
\item
  Against the aggregate of other forecasters. For instance, a
  forecaster's score for a question could be difference between his
  Brier score and the average Brier score of other forecasters who
  participated in the same question. And his total score could be the
  sum of those differences.
\item
  In absolute terms, e.g., a forecaster's score could be their Brier
  score averaged across all questions he has participated in.
\end{itemize}

Both comparison methods do indeed produce a ranking. But they also
introduce a clear zero sum component: forecasters are not incentivized
to share information because that would hurt their relative positioning.
The second method further introduces the distortion that forecasters are
incentivized to choose easier questions, i.e., questions which are
easier to predict on.

The alternative suggested in this paper is to reward forecasters
monetarily in proportion to their improvement over a question's prior
weighted by importance. By the question prior, I mean an initial guess
provided by the question creator or sponsor. And by weighing by
importance, we incentivize forecasters to predict on more important
questions. So overall, we incentivize forecasters to predict on
important questions for which the sponsoring stakeholder's initial guess
is worse.

This creates a more meaningful metric for comparison:
importance-adjusted bits of information added to the decision-maker's
initial forecast. From Bayesian probability theory, we get the strong
hint that probabilities without priors don't really make sense, so that
might be a hint that we want comparisons vis a vis a prior.

There is still a residual incentive to compete with other forecasters to
produce the most importance-adjusted bits of information. However, if
monetary incentives are large enough, the efficiencies through
collaboration and some lower degree of latent altruism might outweigh
those incentives. Conversely, relative positional results could be made
anonymous, by rewarding forecasters using a privacy-preserving
crypto-currency, such as Monero.

\hypertarget{making-forecasters-risk-their-own-funds-makes-them-have-skin-in-the-game}{%
\subsection{Making forecasters risk their own funds makes them have skin
in the
game}\label{making-forecasters-risk-their-own-funds-makes-them-have-skin-in-the-game}}

For a treatment of why this is beneficial, see
\href{https://twitter.com/nntaleb}{Nassim Taleb}'s Incerto. But note
that this step is optional. That is, forecasting platforms could reward
points to forecasters in according to the market scoring rule specified
above. As long as forecaster reward is eventually proportional to points
acquired, this scoring rule becomes proper.

\hypertarget{description-of-the-method.}{%
\section{Description of the method.}\label{description-of-the-method.}}

In the interest of brevity, we shall outline our proposed method by
means of an example, and the example shall be the question ``Will the
People's Republic of China have annexed at least half of Taiwan by
2050?'', as operationalized by
\href{https://www.metaculus.com/questions/5320/chinese-annexation-of-most-of-taiwan-by-2050/}{Metaculus}.
Parts which justify or explain the use of other parts are explained
first, even if they would come later in terms of time.

\hypertarget{the-patron-determines-a-rough-prior-to-reduce-potential-forecasting-reward.}{%
\subsection{The patron determines a rough prior to reduce potential
forecasting
reward.}\label{the-patron-determines-a-rough-prior-to-reduce-potential-forecasting-reward.}}

Taiwan has been independent of mainland China since the 25th of October
1945, i.e., 76 years into the past. Per Laplace's law, the chances that
this will change by 2050 is
\(1-(1-\frac{1}{(2021-1945)+1})^{2050-2021} \approx 32\%\). Lets take
this \(32\%\) as the market's initial probability. Note that per the
\href{https://en.wikipedia.org/wiki/Reference_class_problem}{reference
class problem}, other reference classes might have been chosen, so the
point of this prior is not to be definitive, but rather to provide a
starting point less arbitrary than 50\% from which forecaster reward
might be computed in the next steps.

By sharing this initial step, the question creator saves the time of the
many forecasters which may participate in the question.

In the case of a patron aiming to learn from sponsoring a forecasting
tournament, the prior might represent the patron's initial probability.
If the patron is unsophisticated and the question is not amenable to
base-rate analysis, we may use a 50\%, the percent of questions which
resolve positively on the site, or some other such point. Alternatively,
the patron may sub-contract the creation of the prior, for instance by
paying other forecasters to quickly do so or by creating low-liquidity
(and thus low-potential reward) prediction markets.

If there is no stakeholder or proxy-stakeholder willing and able to
provide a prior, this also provides a useful signal to forecasters that
there might be no stakeholders interested in the question.

\hypertarget{forecasters-predict-on-the-question}{%
\subsection{Forecasters predict on the
question}\label{forecasters-predict-on-the-question}}

Forecasters each get access to a prediction market governed by a
logarithmic market scoring rule, and whose initial probability is the
prior determined in the previous step. This prediction market is also
characterized by a further parameter: liquidity. Liquidity is the amount
of money which is available to take bets on both sides. The more
liquidity there is, the more money it takes to moves a market's odds,
and the more money that can be made if those odds are wrong. For this
reason, question creators would put more liquidity in markets they care
more about. Above, liquidity corresponds to the parameter \(b\).

If an individual forecaster thinks that the initial probability is
wrong, he can move it to a probability which he thinks is more correct
by betting some of his funds. Further, we might hope that the different
forecasters might each research different aspects of the question and
then reveal their information to each other, so that they can
collectively correct their own markets more efficiently. Because each
interacts with his own market. We explore whether this assumption might
break down in Appendix A.

When contemplating this scheme, two issues become apparent after some
amount of reflection. First, in a prediction market in which the outcome
becomes more and more apparent as time goes on, the question creator
loses all the funds he put up as liquidity. Secondly, forecasters have
an incentive to predict as late as possible.

Within the automatic market-maker abstraction, both of these issues have
a clear solution. First, stop trading before the outcome is known. One
particularly advisable point would be to cut trading when the
stakeholder interested in the question makes the decision that the
market was created to influence. Secondly, slowly reduce the amount of
liquidity there is in the markets as time goes on, so that forecasters
have a small incentive to predict sooner rather than later. This is easy
to do by making the parameter \(b=b(t)\), i.e., dependent on time, and
slowly decreasing.

\hypertarget{the-question-gets-a-binary-resolution-and-forecasters-get-rewarded.}{%
\subsection{The question gets a binary resolution and forecasters get
rewarded.}\label{the-question-gets-a-binary-resolution-and-forecasters-get-rewarded.}}

Taiwan either gets invaded or doesn't. Forecasters get rewarded in
proportion to the number of shares of the winning outcome that they
accumulated while interacting with their prediction market.

\hypertarget{the-question-gets-a-probabilistic-resolution}{%
\subsection{The question gets a probabilistic
resolution}\label{the-question-gets-a-probabilistic-resolution}}

Sometimes, the resolution of a question is uncertain. In cases of
uncertainty about how a question should resolve, current forecasting
platforms tend to resolve questions as ambiguous. Prediction markets
tend to break to one side (as 0\% or as 100\%), but they also have the
option of resolving questions probabilistically, so that shares pay out
at \(X\) cts and \((100-X)\) cts rather than \(\$0\) or \(\$1\).

Some common reasons for uncertain resolutions might be:

\hypertarget{because-of-uncertain-ontologies}{%
\subsubsection{Because of uncertain
ontologies}\label{because-of-uncertain-ontologies}}

The Metaculus question as currently written resolves only if the
People's Republic of China takes over at least half of Taiwan. However,
this is somewhat arbitrary. If the question had been on a prediction
market, it could also have been structured such that an invasion of
\(n\)\% of Taiwan pays out \(n\) cents for each share.

\hypertarget{because-of-probabilistic-knowledge}{%
\subsubsection{Because of probabilistic
knowledge}\label{because-of-probabilistic-knowledge}}

At the time of question resolution, Taiwan could be enmeshed in a civil
war, such that it's unclear how much territory each side controls. If
the question had to be resolved, one way to do so would be to pay shares
out in proportion to the likelihood that the PRC controls more than 50\%
of the territory.

\hypertarget{because-the-question-hasnt-resolved-yet}{%
\subsubsection{Because the question hasn't resolved
yet}\label{because-the-question-hasnt-resolved-yet}}

The original question asked about an invasion of Taiwan by 2050, which
is fairly far in the future. So forecasters might not be motivated to
predict in question for which the payoff might be very far away. In the
two accompanying papers, Amplify Samotsvety and Amplify Rootclaim, we
outline two methods for providing forecasters with a reward now for
questions whose resolution is far away. But that reward will be
probabilistic.

That is, shares for the prediction market are paid out according to the
probability estimated by a different forecasting system. If that
forecasting system estimates an X\% probability, shares of yes are paid
out \(X\) cts.

\hypertarget{conclusion}{%
\subsection{Conclusion}\label{conclusion}}

We outlined several improvements to scoring rules as they are currently
implemented in forecasting platforms:

\begin{itemize}
\tightlist
\item
  Instead of comparing forecasters against each other, compare them to
  the initial probability of the stakeholder who is interested in the
  question
\item
  Assign a prediction market governed by an automatic market maker to
  each forecaster, rather than a scoring rule, because this better
  governs how extreme the probabilities which forecasters input can be.
\item
  Make forecasters risk their own funds.
\end{itemize}

These improvements solve the majority of the problems identified in
Sempere and Lawsen's \href{https://arxiv.org/abs/2106.11248}{Alignment
Problems With Current Forecasting Platforms}. Still, the hard work of
implementation is yet to be done. For instance, although we favor the
logarithmic market scoring rule, platforms may find others market
scoring rules more suitable in practice. In addition, our method's cost
increases linearly with the number of forecasters, so platforms should
figure out how to invite only the good ones.

Note also that the three improvements are modular, and their combination
fairly opinionated. But there is nothing preventing platforms from
comparing forecasters' Brier scores against the Brier scores of a
stakeholder, from using a scoring rule or an automatic market-maker
different from the logarithmic
one\footnote{For instance, practitioners such as Gnosis, Augur or [Manifold Markets](https://manifoldmarkets.substack.com/p/above-the-fold-market-mechanics?s=r) have historically prefered the constant product market maker due to ease of implementation},
or from using play-money rather than real-money.

\end{document}
